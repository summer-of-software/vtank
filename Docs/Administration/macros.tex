%%%%%%%%%%%%%%%%%%%%%%%%%%%%%%%%%%%%%%%%%%%%%%%%%%%%%%%%%%%%%%%%%%%%%%%%%%%%
% FILE    : macros.tex
% SUBJECT : File to hold the macros defined in the VTank project.
% AUTHOR  : (C) Copyright 2009 by Vermont Technical College
%
%%%%%%%%%%%%%%%%%%%%%%%%%%%%%%%%%%%%%%%%%%%%%%%%%%%%%%%%%%%%%%%%%%%%%%%%%%%%

% ------------
% New Commands
% ------------

% Add commands in alphabetical order.
\newcommand{\command}[1]{\texttt{#1}}    % For formatting commands.
\newcommand{\filename}[1]{\texttt{#1}}   % For formatting file names.
\newcommand{\newterm}[1]{\textit{#1}}    % For formatting terms when they are first introduced.
\newcommand{\GameServer}{Theater}				 % Holds the game server's name.
\newcommand{\Janitor}{Janitor}					 % Holds the database manager's name.
\newcommand{\MainServer}{Echelon}				 % Holds the main server's name.
\newcommand{\BackupServer}{Silica}       % Holds the backup server's name.
\newcommand{\WebServer}{Porkbarrel}      % Holds the web server's name.
\newcommand{\MapEditor}{Gardener}        % Holds the map editor's name.
\newcommand{\Client}{VTank}
\newcommand{\placeholder}{\textit{placeholder\ldots}}  % To make it easy to find.
\newcommand{\VTank}{VTank}               % For formatting ``VTank.''
\newcommand{\xml}[1]{$<$#1$>$}           % For formatting an XML element inline.
\newcommand{\Launcher}{Launcher}			   % Holds the launcher's name.
\newcommand{\Patcher}{Patcher}			     % Holds the patcher's name.
\newcommand{\FormWrap}{NeoForce}         % Holds the form wrapper used for the client
\newcommand{\AI}{AI Framework}		       % Name of the AI framework.
\newcommand{\WeapProps}{Weapons.xml}             % Name of the xml file for weapon properties.
\newcommand{\ProjProps}{Projectiles.xml}         % Name of the xml file for projectile properties.
\newcommand{\EnvProps}{EnvironmentProperties.xml}         % Name of the xml file for environment properties.

% Versioning
% The following commands encapsulate the versions used for easy editing in the future.
\newcommand{\BoostVersion}{1.39}
\newcommand{\IceVersion}{3.4}
\newcommand{\MiKTeXVersion}{2.7}
\newcommand{\PCLintVersion}{9.0}
\newcommand{\StacklessPythonVersion}{2.6.1}
\newcommand{\TeXnicCenterVersion}{1 Beta 7.50}
\newcommand{\XNAVersion}{3.1}


% ----------------
% New Environments
% ----------------

% An environment to display a sequence of commands.
\newenvironment{commands}
  {\begin{quote} \tt}
  {\end{quote}}


% An environment for displaying use cases.
%   This environment takes three parameters:
%   \param #1: The name of the use case.
%   \param #2: The actor who participates in the use case.
%   \param #3: The context in which the use case executes.
%
%   The body of the environment is the action associated with the use case.
\newsavebox{\UseCaseName}    % Create some boxes to hold the necessary text.
\newsavebox{\UseCaseActor}   % We need to do this because we can't use the
\newsavebox{\UseCaseContext} % environment parameters in the 'end' definition.
\newsavebox{\UseCaseAction}
\newenvironment{usecase}[3]
 {
  \sbox{\UseCaseName}{\bfseries #1}  % The name is easy.
  \sbox{\UseCaseActor}{#2}           % The actor is easy.
  \begin{lrbox}{\UseCaseContext}     % Format the context in a minipage.
    \begin{minipage}{3.25in}
    #3
    \end{minipage}
  \end{lrbox}
  \begin{lrbox}{\UseCaseAction}      % The environment body becomes the action.
  \begin{minipage}{3.25in}}
 {\end{minipage}
  \end{lrbox}
\begin{center}   % Now spew forth the table using the information collected above.
\begin{tabular}{|l||p{3.5in}|} \hline
\multicolumn{2}{|c|}{\usebox{\UseCaseName}} \\ \hline
Actor   & \usebox{\UseCaseActor}   \\ \hline
Context & \usebox{\UseCaseContext} \\ \hline
Action  & \usebox{\UseCaseAction}  \\ \hline
\end{tabular}
\end{center}}


% ------------------
% Various TeX macros
% ------------------

% Macros that allow authors to easily insert initialed comments.

% Chris
\long\def\cbnote#1{\marginpar{CB}{\small \ \ $\langle\langle\langle$\
{#1 -- CB}\
    $\rangle\rangle\rangle$\ \ }} 
    
% Peter
\long\def\pcnote#1{\marginpar{PC}{\small \ \ $\langle\langle\langle$\
{#1 -- PC}\
    $\rangle\rangle\rangle$\ \ }} 
    
% Andy
\long\def\asnote#1{\marginpar{AS}{\small \ \ $\langle\langle\langle$\
{#1 -- AS}\
    $\rangle\rangle\rangle$\ \ }} 
    
% Michael
\long\def\msnote#1{\marginpar{MS}{\small \ \ $\langle\langle\langle$\
{#1 -- MS}\
    $\rangle\rangle\rangle$\ \ }} 
    
% Brian
\long\def\msnote#1{\marginpar{BH}{\small \ \ $\langle\langle\langle$\
{#1 -- BH}\
    $\rangle\rangle\rangle$\ \ }} 
    
% Rob
\long\def\rwnote#1{\marginpar{RW}{\small \ \ $\langle\langle\langle$\
{#1 -- RW}\
    $\rangle\rangle\rangle$\ \ }} 

% Trevor
\long\def\twnote#1{\marginpar{TW}{\small \ \ $\langle\langle\langle$\
{#1 -- TW}\
    $\rangle\rangle\rangle$\ \ }} 
    
% Isaac 
\long\def\ipnote#1{\marginpar{IP}{\small \ \ $\langle\langle\langle$\
{#1 -- IP}\
    $\rangle\rangle\rangle$\ \ }} 
    